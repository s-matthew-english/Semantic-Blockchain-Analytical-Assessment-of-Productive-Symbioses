\thispagestyle{empty}
\vspace*{1.0cm}

\begin{center}
    \textbf{Abstract}
\end{center}

\vspace*{0.5cm}

\noindent

This work presents a compendium of exploratory analyses on the question of possible use cases for the data structure and incentive mechanisms that together form the foundation for the cryptocurrency Bitcoin. 
The contributions are manifold. 
The first of which is a study in the potential for symbioses between principles of linked open data and that of the design principles inherent in the Bitcoin protocol, concluding with an assessment of the primary vectors for mutual amelioration.
Subsequently we detail the motivation and implementation of a ``smart-contract'' programme for blockchain-based certification of academic credentials.
What follows is a proposal for the disintermediation of inter-cryptocurency exchange that adheres to linked open data principles and hastens the coming inter-linked network of disparate blockchains. 
As Bitcoin remains essentially the world's only truly successful blockchain application we present a detailed analysis of the structural characteristics of the Bitcoin blockchain transaction network, specifically in a time of high value fluctuation and extreme volatility. 
Finally we detail a potential framework for the application of Bitcoin design principles to supply chain management. 
Bitcoin and it's associated protocol remain a fertile research frontier. 
The fundamental motivation for the efforts herein described has been to advance understanding of the techniques Bitcoin technology has pioneered and locate them within a broader framework of scientific knowledge.